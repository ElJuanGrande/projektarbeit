\chapter{Analyse und Behebung ausgewählter Usability-Probleme}
\section{Präsentation 2-teiliger Texte in ListCells} \label{sec:analyseListCells}
\textbf{Analyse}

In einer Liste werden Texte angezeigt, die aus zwei Teilen bestehen. In einem Fall ist der erste Teil ein Schlüssel zur Identifizierung eines Fahrzeuges, der zweite Teil ist der Name des Fahrzeuges. Ein Fahrzeugname kann unter Umständen mehr als einmal auftreten, der Fahrzeugschlüssel hingegen ist eindeutig. Während die Schlüssel sich in der Zeichenanzahl nur um zwei bis drei Zeichen unterscheiden, können die Fahrzeugnamen ganz unterschiedliche Längen haben. Derzeit werden die beiden Textbausteine schlicht aneinandergereiht. Dies führt bei längeren Listen schnell zu Unübersichtlichkeit. Daher sollte eine sichtbare Trennung zwischen den beiden Elementen erfolgen.

\begin{figure}[H]
 \centering
 \includegraphics[width=0.3\textwidth]{grafiken/Liste_Beispiel.png}
 \caption{Liste: Ausgangszustand}
 \label{fig:liste1}
\end{figure}

\begin{figure}[H] 
	\centering
	\subfloat[Design 1] {
		\includegraphics[width=0.3\textwidth]{grafiken/Liste_Design1.png}
	}
	\hspace{1.0em}
	\subfloat[Design 2] {
		\includegraphics[width=0.265\textwidth]{grafiken/Liste_Design2.png}
	}
	\hspace{1.0em}
	\subfloat[Design 3] {
		\includegraphics[width=0.3\textwidth]{grafiken/Liste_Design3.png}
	}
	\caption{Design-Vorschläge}
	\label{fig:liste2}
\end{figure}

Alle Design-Vorschläge sind valide Möglichkeiten zur Lösung des Problems, haben jedoch auch ihre Vor- und Nachteile.

Der erste Designvorschlag (Abb. \ref{fig:liste2}(a)) stellt die beiden Teile des Textes sehr gut optisch getrennt dar. Beide Teile sind tabellenartig linksbündig angeordnet. Die ID-Teile aller Texte sind dabei auf einer Höhe, genauso wie die Namensteile. Auf diese Weise entstehen zwei Spalten, in denen die Textbausteine dargestellt werden.

Bei dem zweiten Designvorschlag (Abbildung \ref{fig:liste2}(b)) ist der linke Teil des Textes fett gedruckt, der andere Teil in ganz normalem Stil. Die zweite Lösung setzt zwar die Trennung nicht so sauber wie Vorschlag 1 um, stellt aber so die Fachlichkeit besser dar. Die Fahrzeug ID und der Fahrzeugname sind eine zusammengehörige Einheit. Nach dem Gestaltgesetz der Nähe (Abschnitt \ref{sec:gestaltGesetze}) würden bei Designvorschlag 1 die Fahrzeugschlüssel als eine Objektgruppe und die Fahrzeugnamen als eine Objektgruppe gesehen werden. Da dies nicht die Intention der Abbildung ist, eignet sich die zweite Lösung besser für die Lösung des Problems.

Der dritte Vorschlag gruppiert die Elemente der Liste zwar ebenso wie in Abbildung \ref{fig:liste2}(b), ist jedoch eine eher exotische Darstellungsweise. Hier sind die Elemente an einer gedachten senkrechten Linie zwischen den beiden Teilen angeordnet. Durch die unterschiedliche Bündigkeit entsteht der Eindruck der Gruppierung pro Element, ebenso wie in Vorschlag 2. Durch das Fettdrucken der linken Spalte wird diese dazu hervorgehoben. Das Problem dieser Darstellungsweise ist jedoch der Stilbruch zu den anderen Listendesigns, die im gleichen Bereich angezeigt werden können. Die anderen darstellbaren Listen beinhalten größtenteils keine 2-teiligen Texte. Auf diese Standardvariante der Listen könnte das dritte Design nicht angewandt werden. Es würde daher das Gesetz der Ähnlichkeit (Abschnitt \ref{sec:gestaltGesetze}) verletzt werden und sich so unter Umständen unangenehm auf das Nutzerempfinden auswirken.

\textbf{Behebung}

Die Umsetzung des zweiten Designvorschlages ist nicht so einfach möglich, wie die Abbildung \ref{fig:liste2} zunächst vermuten lässt. Es ist standardmäßig nicht möglich den Text, der in dem Textbereich einer ListCell eingefügt werden soll zu teilen und zwei verschiedene Styles (Per \gls{css}-Klasse) auf die Teile des Textes anzuwenden.

Wenn man nicht die ListCell-Klasse erweitern und stark verändern möchte, bleibt nur die Möglichkeit, über die Graphic, die gesetzt werden kann, einen Text einzufügen. Die Graphic entspricht hier nicht einer Grafik im allgemein gebräuchlichen Sinne, sondern einem \gls{javafx} Node, der eine beliebige Ausprägung haben kann. So könnten in der Theorie auch sehr komplexe Elemente als Listenelemente dargestellt werden.

Der Text, der in den Listenzellen angezeigt werden soll, erhält die Klasse aus dem \gls{dataProvider}. Dieser fügt die Teile der Texte bisher zusammen und liefert sie als einen einzelnen String zurück. Bei der Präsentation werden jedoch Teilstrings benötigt. Nun könnte der Text geparst werden und daraufhin wieder aufgespalten, allerdings wäre dieses Vorgehen nicht sinnvoll, da die gleiche Listenzelle auch für Texte funktionieren soll, die nicht aufgeteilt sind. Möglicherweise würde der Text in dem Fall sogar fälschlicherweise fettgedruckt werden. Aus diesem Grund müssen die Texte von vorneherein als Mehrteiliger Text aus dem \gls{dataProvider} geliefert werden. Die nächstliegende Lösung ist es, ein Array zu benutzen, das die Textteile speichert. Aufgrund der Tatsache, dass die Rückgabewerte aus dem \gls{dataProvider} auch an anderen Stellen verwendet werden, muss sich das zurückgegebene Objekt nach außen hin wie ein String verhalten. Dies kann dadurch erreicht werden, dass das Array in ein neues Objekt verpackt wird, welches die \textit{\#toString()}-, \textit{\#hashCode()}- und \textit{\#equals(Object)}- Methoden überschreibt und die einzelnen Textteile zuvor zu einem String zusammenfügt.

Die Listenzelle muss jetzt nur noch überprüfen, ob es sich bei dem zurückgelieferten Wert um ein Objekt der Klasse \textit{MultiPartString} handelt und kann daraufhin den ersten Teil des Textes anders darstellen.

Für das korrekte Layout wird eine HBox verwendet, deren innerer Abstand zwischen Elementen so eingestellt ist, dass die Teile wie ein einziger Fließtext wirken.
\section{Filter-Performance} \label{sec:analyseFilterPerformance}
Die Änderungen, die in Kapitel \ref{sec:implFilter} eingeführt wurden, funktionieren zwar aus technischer Sicht, lassen den Filter aber bei bestimmten Operationen langsam arbeiten, was zu unbequemen Verzögerungen führt.

\textbf{Analyse}

Um die Laufzeitprobleme beheben zu können, mussten als erstes die Fehler gefunden werden, die zu besagten Problemen führten. Bei der Analyse wurden die folgenden Problemfaktoren untersucht:

\begin{itemize}
	\item Zeitintensive Anweisung (möglicherweise in Schleifen ausgeführt?)
	\item Mehrfach hinzugefügte JavaFX-Listener an ObservableLists
	\item Schleifen mit hohen Durchlaufzahlen
	\item Konstruktion von vielen, großen Objekten
\end{itemize}

Für die Überprüfung des ersten Faktors mussten alle Anweisungen einzeln angeschaut werden. Wenn ein Funktionsaufruf dabei ist, der viele Berechnungen ausführt, kann die Ausführungszeit des Aufrufes dadurch gemessen werden, dass der Zeitpunkt vor der Ausführung gespeichert wird und nach der Ausführung von der dann aktuellen Zeit abgezogen wird. Benutzt dafür den Aufruf System\textit{\#currentTimeMillis()} erhält man die Zeitspanne in Millisekunden. Auf diese Weise ließ sich kein performancekritischer Methodenaufruf feststellen. Auch unter Berücksichtigung, dass manche Methoden in Schleifen ausgeführt werden, fand sich keine kritische Stelle.

Die zweite Überprüfung war erfolgreicher. Tatsächlich wurden zwar keine Listener mehrfach an einer ObservableList angehängt, aber pro Werteliste im Filter (in diesem Fall 13) wurden jeweils einige Listener ausgelöst, die diese aktualisieren sollten, wenn sich die ausgewählten Werte geändert haben. Diese Berechnungen erforderten einigen Aufwand und dadurch, dass sie, statt einmal, 13-mal ausgeführt wurden, sorgten sie für eine negative Laufzeitbeeinflussung.

Diese Lösung dieses Problems alleine brachte jedoch nicht den gewünschten Effekt. Eine Methode, die extrem oft ausgeführt wurde und für die meisten Filterberechnungen relevant ist, ist die Methode, die in der Menge der selektierten Werte sucht, ob eine Attribut-Werte-Kombination vorhanden ist.

\begin{lstlisting}[
    language=Java,
    caption=Aufbau - selectedValues,
    label=code12]
	ObservableList<Pair<AbstractFilterAttribute<A>, Object>> selectedValues;
\end{lstlisting}

Als Parameter erhält die Methode ein Attribut und einen Wert, der überprüft werden soll. Die aufgerufene benutzt die \textit{\#contains(Object)}-Methode der List-Implementierung. Dafür muss für jede Überprüfung ein Objekt vom Typ \textit{Pair} erzeugt werden, das das Attribut und den zu überprüfenden Wert kapselt. Zudem muss danach (durch die Implementierung von \textit{\#contains(Object))} jedes Objekt der Liste durchlaufen werden und per \textit{\#equals(Object)} verglichen werden. Je größer die Menge der ausgewählten Werte wird, desto öfter muss ein solcher Vergleich ausgeführt werden. Die Laufzeit eines Durchlaufes über diese Datenstruktur $O(n)$. Bei einzelnen Aufrufen würde dies nicht weiter problematisch werden, doch durch den neuartigen Filter wird diese Methode nun weitaus häufiger ausgeführt.

\textbf{Behebung: Listenaktualisierung}

Um die Menge an gleichzeitigen Listenaktualisierungen zu reduzieren muss zunächst definiert werden, welche Listen zu welchem Zeitpunkt aktualisiert werden müssen. Das ist zunächst die Liste, die zum aktuellen Zeitpunkt angezeigt wird - die anderen Listen sind dann zwar zeitweise veraltet, dies fällt dem Nutzer jedoch nicht auf, da er die Listen nicht sehen kann. Allerdings muss dann jedes Mal, wenn eine andere Werteliste angezeigt werden soll, diese Liste ebenfalls aktualisiert werden, bevor sie dem Nutzer angezeigt wird.

Das ganze kann nur ermöglicht werden, wenn eine Koppelung mit der Benutzeroberfläche stattfindet. Jedes Mal, wenn der Nutzer ein anderes \gls{attribut} auswählt, um die dazu passende Werteliste zu erhalten, wird eine Methode im Filter ausgeführt, die als Parameter das jetzt aktive \gls{attribut} erhält. Die dazu gehörige Werteliste wird dann einmal erneuert und nur diese durch den Listener aktuell gehalten.

\textbf{Behebung: Zugriff auf selektierte Werte langsam}

Das Problem der Menge der selektierten Werte musste auf Ebene der Datenstruktur angegangen werden. Je nachdem, wie viele Werte bereits selektiert waren, konnten sich die Laufzeiten für die Suche in dieser Struktur auf insgesamt bis zu mehrere Sekunden verlangsamen (kumulierter Gesamtwert aller Ausführungen, die bei einer Neuselektion eines Attributwertes nötig sind).

Für den schnelleren Zugriff wurde eine weitere Datenstruktur eingeführt, die eine andere Repräsentation der selektierten Werte darstellt. Durch eine HashBasedTable, einer Struktur aus der \textit{Google Guava} Bibliothek, war es möglich, die Laufzeitkomplexität für einen Zugriff auf die selektierten Werte von $O(n)$ auf $O(1)$ zu verkürzen. Die Struktur ähnelt einer HashMap, arbeitet jedoch auf zwei Dimensionen und bietet die Möglichkeit, 2 Keys zu verwenden, um einen Wert zu erhalten. Da die Zeitdauer zum Auffinden eines solchen Keys in beiden Dimensionen der HashBasedTable konstant ist, ist ein sehr schneller Zugriff auf Elemente möglich.

\begin{lstlisting}[
    language=Java,
    caption=Datenstruktur - FastSelectedValues,
    label=code16]
	private HashBasedTable<AbstractFilterAttribute<A>, Object, Object> fastSelectedValues;
\end{lstlisting}

In obigem Code-Beispiel wird die Zusammensetzung der neuen Datenstruktur gezeigt. Der erste generische Parameter steht für das Attribut, der zweite für den gesuchten Attributwert. Der dritte Parameter wird nicht verwendet und enthält nur der Boolean-Wert \textit{true}, damit ein Objekt für das Key-Paar vorhanden ist. Die Existenz eines Key-Paares kann dann mit \textit{\#contains(Object, Object)} überprüft werden. Die Laufzeit verringert sich dadurch von anfänglich mehreren Sekunden auf wenige Millisekunden. Um unnötigen Rechenaufwand zu vermeiden, wird die Struktur nur einmal aktualisiert, sobald sich die selektierten Werte verändert haben. Ein weiterer Vorteil ist, dass die Erzeugung von vielen Objekten des Typs \textit{Pair} wegfällt.
\section{\enquote{Quellcode-Usability}} \label{sec:quellcodeUsability}
\textbf{Situation}

Die Java 8 - Sprachfeatures sind ein gutes Mittel, vor Allem in Kombination mit \gls{javafx} und der \gls{streamAPI}, Code prägnant und sehr kurz darzustellen. Doch haben die neuen Features, vornehmlich Lambdas auch die Eigenschaft, Code schnell zu verkomplizieren, wenn sie im Überfluss verwendet werden. Der durch die Definition bestimmte Nachteil eines \gls{lambda}es ist, dass nicht sofort erkennbar ist, welches Interface implementiert wird und was die Parameter zu bedeuten haben. Dies lässt nicht nur ungeübte Java 8 –Programmierer an einigen Codestellen verzweifeln. Ein Konstrukt, das im Quellcode in einer noch ausgeprägteren Variante zu finden war, wird folgend umrissen.

\begin{lstlisting}[
    language=Java,
    caption=Code-Qualität - Gegenbeispiel,
    label=code13]
	boolen hasSubLevels = level.getListEntries().stream().filter(
			entry -> entry instanceof ILevel).count() > 0;
	entryList.setCellFactory(list -> new ListCell<IListEntry>() { 		//1
		{ setPrefWidth(0); } 																//2
		@Override
		protected void updateItem(IListEntry item, boolean empty) {
			super.updateItem(item, empty);
			[...]
			FalkoArrow arrow = new FalkoArrow(Direction.LEFT);
			arrow.setOpacity(item instanceof ILevel ? 1.0 : 0.0); 			//3
			[...]
			HBox hbox = new HBox(5.0);
			if (hasSubLevels) {
				hbox.getChildren().add(arrow);
			}
			[...]
			setOnMouseClicked((v) -> doCallback(item)); 						//4
			setCursor(empty ? Cursor.DEFAULT : Cursor.HAND);
		}
	});
\end{lstlisting}

Der Code ist zwar sehr knapp aber führt selbst mit Syntax-Coloring schnell zur Unübersichtlichkeit. Dieser Effekt wird dadurch verstärkt, dass das selten verwendete Konstrukt des Instanz-Initializers (2), kombiniert mit einer Anonymen Inneren Klasse als Rückgabewert eines Lambdas eingesetzt wird (1). In dieser Klasse werden zusätzlich ternäre Ausdrücke als Alternative zu If-Klauseln (3) und weitere Lambdas (4) benutzt. Zudem hat die Anonyme Innere Klasse Referenzen in den umschließenden Block (Variable \textit{hasSubLevels}).

Ein weiteres Beispiel hängt mit der Verwendung des Double-Colon-Operators zusammen:

\begin{lstlisting}[
    language=Java,
    caption=Code-Qualität - Gegenbeispiel 2,
    label=code14]
	protected StringProperty specialTextProperty = new SimpleStringProperty("[all values]");
	protected final IListEntry specialEntry = specialTextProperty::get;
\end{lstlisting}

Auch über diesen kurzen Code-Ausschnitt kann sich schnell der Kopf zerbrochen werden. Ein \gls{javafx} StringProperty hat nur die Funktion, einen String-Wert zu kapseln und den Zugriff darauf zu erlauben - über die Funktion \textit{\#get()} und \textit{\#set(String)}. Warum ist der obenstehende Code nicht fehlerhaft? Liefert \textit{\#get()} in diesem Falle doch keinen String zurück?
Es sieht in der Tat für ungeübte Java 8 -Entwickler so aus, als würde der Variable \textit{specialEntry} ein String zugewiesen werden. Doch hier handelt es sich um die Kurzschreibweise eines \gls{lambda}es. Die betroffene Zeile würde wie folgt in einen \enquote{echten} Lambda übersetzt werden:

\begin{lstlisting}[
    language=Java,
    caption=Code-Qualität - Verbesserung,
    label=code15]
	protected final IListEntry specialEntry = () -> specialTextProperty.get();
\end{lstlisting}

Um den Sinn genauer verstehen zu können, muss das Interface IListEntry genauer angeschaut werden. Hier befinden sich zwei Methoden. Eine \gls{defaultImpl} für die Methode \textit{\#getText()} und die abstrakte Methode \textit{\#getValue()}, die die String-Repräsentation von \textit{\#getValue()} zurückliefert. Es muss also die Methode \textit{\#getValue()} des funktionalen Interfaces implementiert werden, damit eine vollständige Klasse entsteht. Genau das machen beide die beiden Code-Ausschnitte \ref{code14} und \ref{code15}. Dennoch ist die Schreibweise zunächst unverständlich.

Dies sind nur zwei Beispiele für schwierig lesbaren Code, im Laufe eines Java 8 -Projektes können viele solcher und ähnlicher Probleme auftreten.

\textbf{Lösungsansatz}

Um das erste dargestellte \gls{usability}-Problem zu lösen, kann zum Beispiel die Anonyme Innere Klasse extrahiert werden und als eigenständige Klasse in einer eigenen Datei implementiert werden. Dadurch fiele außerdem der Instance Initializer weg, dessen Code stattdessen im Konstruktor ausgeführt werden kann. Referenzen in den übergeordneten Scope werden dem Konstruktoraufruf als Parameter übergeben.

Das zweite Problem erübrigt sich, wenn auf den \gls{lambda} verzichtet und stattdessen eine Anonyme Innere Klasse implementiert werden würde.

Das Problem der \enquote{Quellcode-\gls{usability}} ist jedoch ein grundlegenderes. Zur Lösung müssen Richtlinien festgelegt werden, wann, in welchem Maße und in welchen Kombinationen Lambdas und andere Sprachfeatures verwendet werden dürfen.

Einige mögliche Richtlinien sind zum Beispiel folgende:

\begin{itemize}
	\item Keine Anonymen Inneren Klassen innerhalb von \glspl{lambda}n verwenden
	\item Lambdas in maximal 2 bis 3 Ebenen schachteln
	\item Bei Schachtelung von Lambdas stets durch Einrückungen kennzeichnen
	\item DoubleColon-Operator nur in selbsterklärenden Fällen verwenden
	\item Parameter in \glspl{lambda}n nach Typ sprechend bezeichnen
\end{itemize}

Die Liste ist weit entfernt davon, komplett zu sein, aber vermittelt dennoch einen kurzen Eindruck, wie Lese- und Verständnisschwierigkeiten im Quellcode vermieden werden können.