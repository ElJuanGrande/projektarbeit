\chapter{Einleitung}
\section{Ziele der Arbeit} \label{sec:einlZiel}
Diese Arbeit hat das Ziel, die Anwendung „FalkoFX“ um einen Anwendungsfall zu erweitern. Dies beinhaltet die Konzeption und die Umsetzung der dafür notwendigen Teillösungen. Damit der Leser einen Eindruck von der Aufgabe und den damit verbundenen Teilproblemen erhält, werden Ausschnitte des bestehenden Konzeptes erläutert. Zur Implementierungen der erarbeiteten Entwürfe werden etablierte Konzepte (Design-Patterns) der Software-Architektur verwendet, die dabei helfen, den Code zu strukturieren und die Programmierarbeit zu erleichtern.
Des Weiteren wird auf einige Probleme in der Gebrauchstauglichkeit der Anwendung eingegangen. Nach einer Analyse dieser, müssen Lösungskonzepte erarbeitet und schlussendlich implementiert werden. Anhaltspunkte für eine sinnvolle Gestaltung bieten unter Anderem allgemeingültige Konzepte, die auf Forschungen der Neuropsychologie basieren.

\section{Motivation} \label{sec:einlMotivation}
Viele bestehende Softwaresysteme sind darauf ausgelegt, zu funktionieren. Das klingt erst einmal nicht verkehrt, aber sollte ein Softwaresystem, das gebraucht wird, nicht auch Gebrauchstauglich sein?
Aufgrund der immer weiter verbreiteten Anforderung der Nutzerfreundlichkeit an bestehende und neue Software, müssen, je nach Komplexität des Systems und dem Budget des Projektes, neue Anwendungen entwickelt werden, die auf dem erlangtem Fachwissen fußen. Teils sind dies komplette Neuentwicklungen, teils Programme mit eingeschränktem Funktionsumfang. Eines haben aber alle 
gemein: Den Fortschritt.

\section{Aufbau der Arbeit} \label{sec:einlAufbau}
Im Rahmen der theoretischen Grundlagen wird zunächst die verwendete Technologie vorgestellt. Es wird ein Einblick in die Neuerungen der Version 8 der Programmiersprache Java gewährt und die große Neuerung, das JavaFX-Toolkit, vorgestellt.
Um die späteren Probleme und deren Lösungen erläutern zu können, wird die Anwendung grob vorgestellt und die Ziele, die mit der Entwicklung dieser Software verfolgt werden, abgegrenzt. In der Arbeit erwähnte Konzepte der Softwarearchitektur werden erklärt und ein Basiswissen für das sinnvolle Design von Benutzeroberflächen wird geschaffen.
Es werden die Teilprobleme der Implementierungsarbeit beschrieben und dem Leser mit Hilfe von Erklärungen, Aufgabenanalyse und Konzepterstellung nähergebracht. Dabei wird auf einige Schwierigkeiten und deren Lösungen in der Implementierung eingegangen.