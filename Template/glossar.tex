\newglossaryentry{falkofx} {
	name=FalkoFX,
	description={Eine Anwendung zur Visualisierung von länderspezifischen Fahrzeugdaten}
}

\newglossaryentry{falko} {
	name=Falko,
	description={Eine Anwendung zur Visualisierung, Verwaltung und Editierung von länderspezifischen Fahrzeugdaten}
}

\newglossaryentry{pattern} {
	name=Design-Pattern,
	description={Etablierte Programmierkonzepte zur Lösung von spezifischen Problemen der Code-Strukturierung}
}

\newglossaryentry{neuropsychologie} {
	name=Neuropsychologie,
	description={Teilgebiet der Psychologie, der sich mit dem Gehirn beschäftigt}
}

\newglossaryentry{usability} {
	name=Usability,
	description={Beschreibt, wie komfortabel ein Softwareprodukt für verschiedene Nutzergruppen bedienbar ist}
}

\newglossaryentry{gebrauchstauglichkeit} {
	name=Gebrauchstauglichkeit,
	description={siehe Usability}
}

\newglossaryentry{javafx} {
	name=JavaFX,
	description={Neuartiges GUI-Toolkit zur Oberflächengestaltung von Java-Anwendungen}
}

\newglossaryentry{defaultImpl} {
	name=Default Implementierung,
	description={Standard-Implementierung für Methoden innerhalb von Interfaces}
}

\newglossaryentry{functionalInterface} {
	name=Functional Interface,
	description={Interface bei dem nur eine Methode implementiert werden muss}
}

\newglossaryentry{stream} {
	name=Stream,
	description={Datenstruktur, die Collection-Datentypen kapselt und komplexe Operationen darauf ermöglicht, ohne Schleifen manuell ausprogrammieren zu müssen}
}

\newglossaryentry{streamAPI} {
	name=Stream API,
	description={Programmierschnittstelle zu Verarbeitung von Streams}
}

\newglossaryentry{lambda} {
	name=Lambda-Ausdruck,
	plural=Lambda-Ausdrücke,
	description={Kurzschreibweise für Anonyme Innere Klassen}
}

\newglossaryentry{css} {
	name=CSS,
	description={Cascading Style Sheets: Beschreibungssprache für visuelle Präsentation von UI-Elementen}
}

\newglossaryentry{gui} {
	name=GUI,
	description={Graphical User Interface: \textit{Benutzeroberfläche}}
}

\newglossaryentry{property} {
	name=Property,
	plural=Properties,
	description={Beschreibende Eigenschaft eines Objektes, z.B. in Form einer Membervariablen des Typs ObservableValue}
}

\newglossaryentry {observable}{
	name=ObservableValue,
	description={Wert eins beliebigen Typs, dessen Änderungen per JavaFX-Listener publiziert werden können}
}

\newglossaryentry{binding} {
	name=Binding,
	description={Bindet zwei ObservableValues aneinander - bei Veränderung des Einen Wertes wird der Andere mit verändert}
}

\newglossaryentry{gallery} {
	name=Galerie,
	description={Selbstentwickelte Komponente zu anschaulichen Präsentation von Daten}
}

\newglossaryentry{mvc} {
	name=MVC-Pattern,
	description={Model-View-Controller-Pattern: Design-Pattern zur Einteilung des Codes in drei Zuständigkeitsbereiche}
}

\newglossaryentry{commandPattern} {
	name=Command-Pattern,
	description={Design-Pattern zum Ausführen von Methoden zu einem beliebigen Zeitpunkt nach Erzeugung eines \textit{Commands}}
}

\newglossaryentry{rohdaten} {
	name=Rohdaten,
	description={Aus der Datenbank geladene, unverarbeitete Daten}
}

\newglossaryentry{filterattribut} {
	name=Filterattribut,
	description={Attribut zur Gruppierung von Filterwerten im Filter}
}

\newglossaryentry{filtervalue} {
	name=Filterwert,
	description={Wert zur Berechnung der Ergebnismenge im Filter}
}

\newglossaryentry{filter} {
	name=Filter,
	description={Konstrukt zum Einschränken und Verarbeiten der rohdaten zu einer Ergebnismenge}
}

\newglossaryentry{ergebnismenge} {
	name=Ergebnismenge,
	description={Menge der Modelobjekte, die nach Filterung übrig geblieben sind}
}

\newglossaryentry{dataProvider} {
	name=DataProvider,
	description={Java-Klasse zum Bereitstellen und Laden der Rohdaten}
}

\newglossaryentry{attribut} {
	name=Attribut,
	description={siehe Filterattribut}
}

\newglossaryentry{valueResolver} {
	name=ValueResolver,
	description={Java-Klasse zum Ermitteln von Werten für ein Attribut zu einem Modelobjekt}
}

\newglossaryentry{filterElement} {
	name=FilterElement,
	description={Pseudo-Modelobjekte, um ein Land und ein Fahrzeug für eine Freigabe zu k apseln}
}

\newglossaryentry{freigabe} {
	name=FreigabeObjekt,
	description={Modelobjekt der Produktionsfreigabe}
}

\newglossaryentry{getter} {
	name=Getter,
	description={Funktion, die den Wert einer Membervariablen zurückgibt}
}

\newglossaryentry{setter} {
	name=Setter,
	description={Funktion, die den Wert einer Membervariablen setzt}
}

\newglossaryentry{multiFilter} {
	name=MultiFilter,
	description={Filter für den 3. Anwendungsfall; Beinhaltet zwei separate Filter}
}