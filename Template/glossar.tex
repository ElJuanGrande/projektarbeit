\newglossaryentry{swipe} {
name=Swipe-Geste,
	description={Wischbewegung auf einem Touchbildschirm.}
}
\newglossaryentry{use} {
name=Usebillity,
	description={\editHere{TODO}}
}
%Erklaeren: Domain Object, SVG, Overlay, Datenstand (müsste bereits erklärt werdne), Plandaten, setGraphic-Methode bei TableColumns %Widget (WindowGadget: Fenster, das Eingabeereignesse empfängt, Widgets sind immer in ein bestimmtes Fenstersystem eingebunden und nutzen dies zur Interaktion mit dem Anwender oder anderen Widgets)
%GUI = Benutzeroberfläche

\newglossaryentry{ergonomisch} {
	name=ergonomisch,
	description={Bedeutung laut Duden: \glqq die Ergonomie betreffend, auf den Erkenntnissen der Ergonomie beruhend\grqq}}

\newglossaryentry{anfErheber} {
	name=Anforderungserheber,
	description={Ann-Katrin Hannemann, die die Anforderungen ermittelt und dokumentiert hat}}

\newglossaryentry{grauzone} {
	name=Grauzone,
	description={Unbekannter Bereich eines \gls{systemkontext}es. Die Systemgrenze kann aufgrund fehlender Informationen nicht vollst\"andig bestimmt werden}}
	
\newglossaryentry{systemkontext} {
	name=Systemkontext,
	description={Der f\"ur die Definition und das Verst\"andnis der Anforderungen entscheidende Teil der Umgebung eines betrachteten Systems }}

\newglossaryentry{quellen} {
	name=Quellen,
	description={Ursprung von zum Beispiel Daten- und Materialfl\"ussen}}
	
\newglossaryentry{senken} {
	name=Senken,
	description={Ausgaben des Systems. Sie sind die Endpunkte der Datenfl\"usse zwischen dem System und seiner Umgebung}}

\newglossaryentry{systemgrenze} {
	name=Systemgrenze,
	description={Trennung von Bestandteilen des Systems, die sich innerhalb der Systemgrenze befinden, und den anderen Teilen des \gls{systemkontext}es}}
	
\newglossaryentry{kontextgrenze} {
	name=Kontextgrenze,
	description={Befasst sich mit den Beziehungen zwischen Aspekten des geplanten Systems und relevanten Aspekten der Umgebung}}
	
\newglossaryentry{pki} {
	name=PKI-Karte,
	description={Karte, die zur elektronischen Identifizierung notwendig ist}}
	
\newglossaryentry{langStraPlan} {
	name=langfristige strategische Produktplanung,
	plural=langfristigen strategischen Produktplanung,
	description={\"Uberlegungen bez\"uglich best\"andiger Auswirkungen von Produktionsfaktoren beim Herstellen von \gls{fzg}en (Produkten) unter Ber\"ucksichtigung von Konzernstrategien}}

\newglossaryentry{zeitskala} {
	name=Zeitskala,
	description={Angabe eines Zeitraumes, f\"ur den die \gls{report}s erstellt werden sollen, bei den Eigenschaften von \gls{cypris}}}

\newglossaryentry{bx} {
	name=\mbox{BREDEX GmbH},
	description={Eine auf Softwareentwicklung und Beratung spezialisierte Firma, an der die in diesem Dokument beschriebene Anforderungserhebung f\"ur einen Kunden durchgef\"uhrt wird}}
	
\newglossaryentry{planGremium} {
	name=Gremium,
	description={Gremium  zur Produktplanung: Eine Gruppe von Personen, die sich mit dem zentralen Anliegen der \gls{prod}(Fahrzeug-)planung f\"ur den \gls{vwkonzern} befasst},
	plural=Gremien}
	
\newglossaryentry{planungsrunde} {
	name=Planungsrunde,
	description={J\"ahrliches Treffen des Gremiums zur Produktplanung bei dem die Fahrzeuge des gesamten \gls{vwkonzern}s grob geplant werden}}

\newglossaryentry{isso} {
	name=ISSO,
	description={Sicherheitsabteilung der \gls{vwag}}}

\newglossaryentry{prod} {
	name=Produkt,
	description={In dieser Arbeit ein Synonym f\"ur \gls{fzg}e (siehe \gls{fzg})}}

\newglossaryentry{fzg} {
	name=Fahrzeug,
	description={Kraftfahrzeug, verschiedene Arten von Automobilen}}

\newglossaryentry{fzgattr} {
	name=Fahrzeugeigenschaft,
	description={Attribute zu einem geplanten \gls{fzg}. Die f\"ur das Projekt relevanten Fahrzeugeigenschaften sind: \gls{marke}, \gls{seg}, \gls{generation}, \gls{bs}, \gls{fertigungsregion} und \gls{antriebsart}},
	plural=Fahrzeugeigenschaften}
	
\newglossaryentry{terminattr} {
	name=Termineigenschaft,
	description={Unter anderem eine \gls{terminart} oder ein Datum, an dem eine bestimmte \gls{terminart} vorgesehen ist},
	plural=Termineigenschaften}
	
\newglossaryentry{terminart} {
	name=Terminart,
	description={Eine Termineigenschaft, die folgende Werte haben kann: Produktionsstart, -ende oder \gls{modellpflege}}}

\newglossaryentry{prodplaner} {
	name=Produktplaner,
	description={Verwalten mit Hilfe von \gls{cypris} die geplanten Fahrzeuge und Termine}}
	
\newglossaryentry{refdaten} {
	name=Referenzdaten,
	description={Definierte Werte für Fahrzeug- und Termineigenschaften.}}
	
\newglossaryentry{vwkonzern} {
	name=Volkswagen Konzern,
	description={Einer der f\"uhrenden Automobilhersteller weltweit. Zu ihm geh\"oren die folgenden \gls{marke}n: Volkswagen Pkw, Audi, SEAT, SKODA, Bentley, Bugatti, Lamborghini, Porsche, Ducati, Volkswagen Nutzfahrzeuge, Scania und MAN. Der Volkswagen Konzern befasst sich au\ss erdem noch mit anderen Gesch\"aftsfeldern und bietet zum Beispiel auch Finanzdienstleistungen an}}

\newglossaryentry{vwag} {
	name=Volkswagen AG,
	description={siehe \gls{vwkonzern}}}

\newglossaryentry{cypris} {
	name=\mbox{CYPRIS},
	description={Kurzform f\"ur Cycle Plan Pr\"asentations- und Informationssystem. Eines der Softwaresysteme, die in der \gls{vwag} bei der Produktplanung unterst\"utzten. Es wird speziell f\"ur den langfristigen strategischen Fahrzeugplanungsprozess eingesetzt}}
	
\newglossaryentry{cyprisDesktop} {
	name=\gls{cypris} Desktop Client,
	description={\gls{cypris} Desktopanwendung}}

\newglossaryentry{csr} {
	name=CSR,
	description={Kurzform f\"ur \gls{cypris} Simple Reporting}}

\newglossaryentry{csrClient} {
	name=CSR Client,
	description={CSR Anwendung}}


\newglossaryentry{datenaufbereitung} {
	name=Datenaufbereitung,
	description={Notwendiger Anwendungsfall zur Erstellung eines \gls{report}s mit dem \gls{cyprisDesktop}. Der Anwender legt dabei Parameter fest, um zu bestimmen welche Informationen in den Bericht sollen. Diese Parameter geben den Datenstand und die \gls{datenbasis}, aus denen der \gls{report} generiert werden soll, an}}
	
\newglossaryentry{datenbasis} {
	name=Datenbasis,
	description={Menge der Fahrzeugdaten aus der Datenbank}}
	
\newglossaryentry{datenpflege} {
	name=Datenpflege,
	description={Verwaltung der Daten aus der Datenbank. Dabei k\"onnen Daten erstellt, bearbeitet und gel\"oscht werden}}

\newglossaryentry{layoutanpassung} {
	name=Layoutanpassung,
	description={Notwendiger Anwendungsfall zur Erstellung eines \gls{report}s mit dem \gls{cyprisDesktop}. Dabei bestimmt der Anwender wie der \gls{report} aussehen soll}}

\newglossaryentry{3schichtArch} {
	name=Drei-Schichten-Architektur,
	description={Strukturierungsprinzip der Software-Architektur mit drei Schichten: der Pr\"asentationsschicht, der Logikschicht und der Datenerhaltungsschicht}}

\newglossaryentry{modellpflege} {
	name=Modellpflege,
	description={Der Begriff stammt aus der Automobil- und Motorradbranche. Er bezeichnet die optische und technische \"Uberarbeitung eines Fahrzeugmodells}
}

\newglossaryentry{datenstand}{
	name=Datenstand,
	description={Ein definierter Status eine Menge von \gls{plandaten}},
	plural=Datenst\"ande}
	
\newglossaryentry{plandaten}{
	name=Plandaten,
	description={Termine und \gls{fzg}e, die mittels grober Eigenschaften beschrieben werden. In der {planungsrunde} zur Produktplanung werden die \gls{fzg}eigenschaften besprochen und ihnen Termineigenschaften zugewiesen}}
	
\newglossaryentry{arbeitsstand}{
	name=Arbeitsstand,
	description={Ein \gls{datenstand} in \gls{cypris}, der noch bearbeitet wird und nicht von einem \glspl{planGremium} best\"atigt wurde},
	plural=Arbeitsst\"ande}

\newglossaryentry{freigegebenerDatenstand}{
	name=freigegebener Datenstand,
	description={Ein von \glspl{planGremium} \"uberpr\"ufter und best\"atigter \gls{datenstand}}}
	
\newglossaryentry{marke} {
	name=Marke,
	description={Der Begriff steht f\"ur alle Eigenschaften, zur Unterscheidung zur Konkurrenz, die mit einem Markennamen in Verbindung stehen. In dieser Arbeit die Automarken des \gls{vwkonzern}s}}

\newglossaryentry{herstellermarkt}{
	name=Herstellermarkt,
	description={Siehe \gls{fertigungsregion}}}

\newglossaryentry{markt} {
	name=Markt,
	description={L\"ander bzw. L\"andergruppe in die ein Produkt verkauft wird/werden soll}}

\newglossaryentry{seg} {
	name=Fahrzeugklasse,
	plural=Fahrzeugklassen,
	description={Ein anderes Wort f\"ur Segment (Bezeichnung im VW Umfeld), auch Wettbewerbsklasse genannt. \gls{fzg}e desselben Segments k\"onnen im gesamten Konzern der \gls{vwag} miteinander verglichen werden. Ein Beispiel f\"ur ein Segment ist die A-Klasse}}
	
\newglossaryentry{generation} {
	name=Generation,
	description={Viele \gls{fzg}e werden \"uber mehrere Jahrzehnte weiter entwickelt. Bei gro\ss en \"Anderungen, also nicht nur einer \gls{modellpflege}, wird die Generationsnummer eines Fahrzeuges hochgez\"ahlt. So folgte auf den Golf IV der Golf V}}
	
\newglossaryentry{bs} {
	name=Karosserieform,
	plural=Karosserieformen,
	description={Auch Bodystyle genannt. Beschreibt verschiedene Fahrzeugtypen nach ihrer Bauart. Zum Beispiel Kurzheck, Stufenheck, Kombi, Flie{\ss}heck/Sportback, Coupe, Cabrio/Roadster, SUV, Stadtlieferwagen/ Pick-up, MPV, Sonst. Bauart eines \gls{fzg}es \editHere{Verweis auf Kapitel Systembeschreibung CYPRIS}}}
	
\newglossaryentry{fertigungsregion} {
	name=Fertigungsregion,
	description={Synonym f\"ur \gls{herstellermarkt}. Sie beschreibt das Land oder die L\"andergruppe, in der das \gls{fzg} produziert wird. Wertebeispiele dieser Eigenschaft sind zum Beispiel die EU, China, Nordamerika, Mexiko}}
	
\newglossaryentry{antriebsart} {
	name=Antriebsart,
	description={Zum Beispiel: Konventionell, Hybridelektrokraftfahrzeug (HEV), Plug-in-Hybridelektrokraftfahrzeug(PHEV), Batterie-Elektrokraftfahrzeug (BEV), Erdgas (CNG), Autogas (LPG)}}
	
\newglossaryentry{report} {
	name=Report,
	description={Ein Bericht. Es gibt verschiedene Reportarten die mit \gls{cypris} erstellt werden k\"onnen: Cycle Plan, Tafelberg, Produktportfolio, Jahresanl\"aufe}}
	
\newglossaryentry{ppf} {
	name=Produktportfolio,
	description={Palette von \gls{fzg}en (Produktobjekten)}}

\newglossaryentry{ppfreport} {
	name=Produktportfolioreport,
	description={Abbildung eines \gls{ppf}s in Form einer Tabelle als \gls{report}}}

%Kanomodell
\newglossaryentry{kano} {
	name=Kano-Modell,
	description={Das Kano-Modell ist eine M\"oglichkeit Anforderungen zu Kategorisieren. Dies kann nach \gls{kanobasis}, \gls{kanoleistung} und \gls{kanobegeisterung} geschehen. Zus\"atzlich gibt es Varianten mit weiteren Kategorisierungen nach \gls{kanounerheblich} und \gls{kanorueckweisung}}}

\newglossaryentry{kanobasis} {
	name=Basisfaktoren,
	description={Anforderungen, die f\"ur den Anwender selbstverst\"andlich sind. Er kennt diese Anforderungen nicht alle bewusst. F\"ur ihn sind die Basisfaktoren selbstverst\"andlich und f\"uhren daher bei nicht Erf\"ullung zu Unzufriedenheit, werden aber meist nur implizit gefordert}}
	
\newglossaryentry{kanoleistung} {
	name=Leistungsfaktoren,
	description={Anforderungen, die bei deren Einhaltung der geforderten Eigenschaften f\"ur Zufriedenheit sorgen}}
	
\newglossaryentry{kanobegeisterung} {
	name=Begeisterungsfaktoren,
	description={Anforderungen, die f\"ur eine Neuheit sorgen. Sie \"uberraschen den Kunden positiv und k\"onnen f\"ur die Kaufentscheidung entscheidend sein}}
	
\newglossaryentry{kanounerheblich} {
	name=unerhebliche Faktoren,
	description={Anforderungen, die ohne Belang sind}}
	
\newglossaryentry{kanorueckweisung} {
	name=R\"uckweisungsfaktoren,
	description={Anforderungen, die beim Kunden f\"ur Unzufriedenheit sorgen und Schuld daran sind, wenn er zur Konkurrenz geht, wenn sie nicht erf\"ullt werden.}}

%Anforderungen
\newglossaryentry{anforderung} {
	name=Anforderung,
	description={Festlegung, was das geplante System erf\"ullen muss. Dazu geh\"oren zum Beispiel Funktionalit\"at, Qualit\"at und Aussehen}}
	
	%Anforderungsarten
\newglossaryentry{funkAnf} {
	name=funktionale Anforderungen,
	description={Eine Anforderungsart, die die Funktionalit\"at, welche das System haben soll, festlegt},
	plural=funktionalen Anforderungen}

\newglossaryentry{nfAnf} {
	name=nicht funktionale Anforderungen,
	description={Synonym f\"ur \gls{qualiAnf}.}}

\newglossaryentry{qualiAnf} {
	name=qualitative Anforderungen,
	description={Eine Anforderungsart, die die qualitativen Eigenschaften von Funktionen des betrachteten Systems oder dieses Systems beschreiben. In der Literatur werden sie h\"aufig als \gls{nfAnf} bezeichnet.	Zu ihr geh\"oren Wartbarkeits- und Laufzeitanforderungen. Typische qualitative Anforderungen
beschreiben die Performance, die Verf\"ugbarkeit, die Zuverl\"assigkeit, die Skalierbarkeit oder die Portabilit\"at eines Systems.}}

\newglossaryentry{rahmenBed} {
	name=Rahmenbedingungen,
	description={Werden auch Randbedingungen genannt. Die einzige Anforderungsart, die Einschr\"ankungen zur Realisierung des Systems beschreibt. Dies k\"onnen organisatorische oder technische Vorgaben sein. Ein Beispiel w\"are die Wahl des Betriebssystems}}
		
	%Anforderungsquellen
\newglossaryentry{stakeholder} {
	name=Stakeholder,
	description={Eine Anforderungsquelle, die sowohl Personen, Gruppen und Organisationen sein kann und durch den Einsatz oder Betrieb des Systems betroffen ist und ihn direkt oder indirekt beeinflussen}}
	
\newglossaryentry{prodOwner} {
	name=Product-Owner,
	description={Ein Stakeholder, der durch den Projektleiter von \gls{cypris} Alexander Preuk vertreten wird}}
	
\newglossaryentry{anwAktuell} {
	name=aktuelle Anwender,
	description={Eine Stakeholdergruppe bestehender Benutzer der Desktop Anwendung \gls{cypris}. Sie verf\"ugt \"uber das notwendige Expertenwissen zur \gls{report}generierung}}
	
\newglossaryentry{anwNeu} {
	name=zuk\"unftige Anwender,
	description={Eine Stakeholderguppe mit Personen aus den \glspl{planGremium}. Sie k\"onnen aktuelle selbst keine \gls{report}erstellen, sondern beantragen sie bei der Stakeholdergruppe \glqq\gls{anwAktuell}\grqq}}

\newglossaryentry{entwickler} {
	name=Entwickler,
	description={Ein Stakeholder, der durch die Bachelorandin Ann-Katrin Hannemann vertreten ist, die das Konzept f\"ur den CSR zum Erstellen eines Produktportfolioreportes erstellt und diesen daraufhin implementieren wird}}	

\newglossaryentry{hacker} {
	name=Hacker,
	description={Ein Stakeholder der versucht dem System zu schaden. Diese Gruppe wird durch Sicherheitsabteilung \gls{isso} vertreten, die das Vereiteln von Hackerangriffen ist.}}	
	
\newglossaryentry{fachExp} {
	name=Experte f\"ur Fachlichkeit,
	plural=Experten f\"ur Fachlichkeit,
	description={Ein Stakeholder, der durch den Anforderungsmanager von \gls{cypris}, Mathias Leiner, vertreten ist}}	

\newglossaryentry{techExp} {
	name=technische Experte,
	plural=technischen Experten,
	description={Zwei verschiedene Stakeholdergruppen, Experten f\"ur \gls{cypris} und Experten f\"ur mobile Anwendungen. Zu den Experten f\"ur \gls{cypris} z\"ahlen Mitarbeiter der \gls{bx}, die seit mehreren Jahren in dem Projekt \gls{cypris} arbeiten und sich am besten mit diesem System auskennen. Experte f\"ur mobile Anwendungen ist durch jemanden vertreten, der sich mit einigen Besonderheiten bei der Umsetzung mobiler Anwendungen auskennt}}	

\newglossaryentry{designExp} {
	name=Experte f\"ur Oberfl\"achengestaltung,
	description={Ein Stakeholder, der durch eine gelernte Mediengestalterin vertreten wird und auf ein brauchbares Bedienkonzept achtet}}	
	
%Ermittlungstechniken
\newglossaryentry{befragung} {
	name=Befragungstechnik,
	plural=Befragungstechniken,
	description={Mit ihnen kann explizites Wissen von Stakeholdern ermittelt werden. Die Anforderungen werden direkt vom Stakeholder erfasst und sind daher, im Vergleich mit anderen Ermittlungsarten, genauer und unverf\"alscht. Die beiden bekanntesten Arten dieser Technik sind das \gls{interview} und Befragung mit \glspl{fragebogen}}}
	
\newglossaryentry{interview} {
	name=Interview,
	description={Eine \gls{befragung} zu der jede Art der m\"undlichen Befragung z\"ahlt. Diese Technik kann als Einzel- oder Gruppengespr\"ach durchgef\"uhrt werden}}
		
\newglossaryentry{fragebogen} {
	name=Fragebogen,
	description={Eine schriftliche \gls{befragung}, mit vorgegebenen Antwortm\"oglichkeiten oder offenem Antworttext},
	plural=Frageb\"ogen}

\newglossaryentry{kreativTech} {
	name=Kreativit\"atstechnik,
	plural=Kreativit\"atstechniken,
	description={Eine Ermittlungstechnik bei der die Beteiligten ihrer Kreativit\"at freien Lauf lassen k\"onnen}}
	
\newglossaryentry{brainstorming} {
	name=Brainstorming,
	description={Eine \gls{kreativTech}, bei der alle Ideen notiert werden}}
	
\newglossaryentry{brainstormingParadox} {
	name=Brainstorming paradox,
	description={Eine \gls{kreativTech} die der Technik des \gls{brainstorming}s \"ahnlich ist. Im Gegensatz zum \gls{brainstorming}, werden hierbei Ideen welche Ereignisse verhindert werden sollen gesammelt und Gegenma\ss nahmen \"uberlegt}}

\newglossaryentry{persWechsel} {
	name=Perspektivenwechsel,
	description={Eine \gls{kreativTech}, bei der ein Problem von verschiedenen Seiten betrachtet wird. Es eignet sich besonders gut f\"ur komplexe Probleme}}
		
\newglossaryentry{analogie} {
	name=Analogietechnik,
	description={Eine \gls{kreativTech}, bei der nach \"ahnlichen Problemstellungen oder Systemen gesucht wird}}

\newglossaryentry{dokuTech} {
	name=dokumentenzentrierte Techniken,
	plural=dokumentenzentrierten Techniken,
	description={Wiederverwendung existierender Anforderungen von bestehenden Systemen}}
	
\newglossaryentry{systemAch} {
	name=Systemarch\"aologie,
	description={Eine \gls{dokuTech}, bei der Informationen aus der Dokumentation oder Implementierung von Alt- oder Konkurrenzsystem gewonnen werden}}
	
\newglossaryentry{persLesen} {
	name=perspektivenbasiertes Lesen,
	plural=perspektivenbasierte Lesen,
	description={Eine \gls{dokuTech}, bei der ein Leser eine bestimmte Perspektive einnimmt und nur die f\"ur diese Perspektive relevanten Informationen beachtet}}

\newglossaryentry{Wiederverwendung} {
	name=Wiederverwendung,
	description={Eine \gls{dokuTech}, bei der die Kosten der Anforderungsermittlung wesentlich reduzieren k\"onnen, da bereits zuvor erfasste Anforderungen verwendet werden}}
		
\newglossaryentry{beobachtung} {
	name=Beobachtungstechniken,
	description={Diese Technik zur Anforderungserhebung wird eingesetzt, 
wenn Mitarbeiter zwar Fachwissen haben, dieses aber nicht sprachlich ausdr\"ucken k\"onnen oder keine Zeit haben bei der Anforderungsermittlung mitzuarbeiten. Anforderungserheber beobachten sie dabei bei ihrer Arbeit und notieren sich die Arbeitsschritte}}

\newglossaryentry{feldbeobachtung} {
	name=Feldbeobachtung,
	description={Eine der \gls{beobachtung} bei der die Stakeholder am Arbeitsplatz bei Prozessen, Handgriffen und Arbeitsabl\"aufen beobachtet werden}}
	
\newglossaryentry{apprenticing} {
	name=Apprenticing,
	description={Eine der \gls{beobachtung} bei der ein Stakeholder dem Anforderungserheber die T\"atigkeiten lehrt}}
	
\newglossaryentry{unterstTech} {
	name=unterst\"utzende Techniken,
	description={Hilfstechniken f\"ur Techniken zur Anforderungsermittlung. Beispiel sind die Darstellung von Anwendungsabl\"aufen, Prototypen, Audio- und Videoaufzeichnung, CRC-Karten (Class Reponsibility Collaboration), Workshops und Mindmapping}}
	
\newglossaryentry{bestSys} {
	name=bestehende Systeme,
	description={Konkurrenzsysteme, Alt- oder Vorg\"angersysteme}}